\documentclass[runningheads,a4paper]{llncs}

% Packages
\usepackage{amssymb}
\setcounter{tocdepth}{3}
% Our packages
\usepackage{graphicx} % Deal with images
\usepackage[utf8]{inputenc} % Provide special ponctuations
\usepackage{subcaption}
\captionsetup{compatibility=false}
\renewcommand{\thesubfigure}{\roman{subfigure}}
\usepackage{paralist}
\usepackage{url}
\usepackage{amsmath}
% Algorithm
\usepackage{algorithm}
\usepackage[noend]{algpseudocode}
\algrenewcommand\Return{\State \algorithmicreturn{} }
\algnewcommand{\LineComment}[1]{\State \(\triangleright\) #1}

%% Emails
%\urldef{\mailsa}\path|{butz,dossantos,oliveira}@cs.uregina.ca|
\urldef{\mailsa}\path|{author1,author2,author2}@institute1|
\newcommand{\keywords}[1]{\par\addvspace\baselineskip
\noindent\keywordname\enspace\ignorespaces#1}

\begin{document}

% Title
\mainmatter
\title{On Theoretical Properties of \\ Augmented Sum Product Networks}
\titlerunning{On Theoretical Properties of \\ Augmented Sum Product Networks}

%% Authors
%\author{
%Andr\'e E. dos Santos\inst{1} 
%\and
%Cory J. Butz\inst{1} 
%\and 
%Jhonatan S. Oliveira\inst{1} 
%}
% Authors
\author{
Author 1\inst{1} 
\and
Author 2\inst{1} 
\and 
Author 3\inst{1} 
}
\authorrunning{On The Conversion of Augmented Sum Product Networks to Bayesian Networks}

% Institution
%\institute{
%Department of Computer Science, University of Regina, Canada\\
%\mailsa\\
%}
\institute{
Department 1, Institute 1, Country 1\\
\mailsa\\
}

% Print title
\toctitle{On The Conversion of Augmented Sum Product Networks to Bayesian Networks}
\tocauthor{On The Conversion of Augmented Sum Product Networks to Bayesian Networks}
\maketitle


% Abstract

\begin{abstract}
\emph{Sum-Product Network} (SPN) is probabilistic graphical model that also benefits from the properties of deep learning models.
%In an SPN, leaf nodes are indicators over Boolean variables and the remaining variables are either sum or product nodes.
The main feature in SPNs is that inference is linear with respect to the size of the network when certain constrains are satisfied.
There are two methods in the literature to convert an SPN to a \emph{Bayesian Network} (BN).
The first method applies the conversion of SPNs into BNs using algebraic decision diagrams to represent the local conditional probability distributions.
The second method is given by the theoretical foundation for latent variable interpretation in augmented SPNs.
In this paper we propose a method for conversion of augmented SPN to BNs.
Our method can convert augmented SPNs to BNs preserving the dependencies among random variables and latent variables.
%The introduced nodes in an augmented SPN is dealt without 
As results, we establish a common ground between the current methods for converting SPNs to BNs.
We also propose an method for augmenting an SPN that requires less modifications on the graph.
Our work continues to give a broader perspective of probabilistic semantics implied by the structure of an SPN through generated the BN.
\end{abstract}

\keywords{sum product networks, conditional independence, deep learning}


% Sections
\section{Introduction}

[Background]\\
Two methods for augmenting SPNs.\\
Poupart and Pehartz.\\
Pehartz is better because it fixes the conflicting of indicator variables as LV with the completeness condition.\\
Also, Pehartz proposes a LV interpretation which allows to see augmented SPNs as BNs.\\
This resulted in a sound derivation of EM for SPNs.


[Problem] \\
The augmentation was proposed as a theoretical too to establish an work with the LV interpretation in SPNs
The augmented SPN can grow quadratically in $|S|$ for upper bound.

[New augmentation]

[How new augmentation affects the CIs]

[How new augmentation affects the EM] \\
Future work
\section{Background}

\subsection{Required Notation}

We first define some pertinent notation used in this paper.
A random variables is denoted by upper-case letter $X$.
The values of $X$ are denoted by $val(x)$, where lower-case element $x$ is a corresponding element of $val(X)$.
Sets of random variables are defined by boldface letter ${\bf X}$.
For ${\bf X} = \{X_1,\dots,X_N\}$, we define ${\bf val(X)} = \times^{N}_{n=1}{{\bf val}(X_n)}$ and use corresponding lower-case boldface  letters for elements of ${\bf val(X)}$.
For a subset ${\bf Y \subseteq X}$, ${\bf x[Y]}$ is a projection of ${\bf x}$ onto ${\bf Y}$.

Let ${\cal G}$ denote a \emph{directed acyclic graph} (DAG) on a finite set of variables (nodes).
%A \emph{directed path} from $v_1$ to $v_k$ is a sequence $v_1, v_2, \ldots , v_k$ with directed edges $(v_i, v_{i+1})$ in ${\cal B}$, $i = 1, 2,\ldots, k-1$.
%For each $v_i \in U$, the \emph{ancestors} of $v_i$, denoted $An(v_i)$, are those variables having a directed path to $v_i$.
%For a set $X \subseteq U$, we define $An(X)$ in the obvious way.
The \emph{children} $Ch(v_i)$ and \emph{parents} $Pa(v_i)$ of $v_i$ are those $v_j$ such that $(v_i,v_j) \in {\cal G}$ and $(v_j,v_i) \in {\cal G}$, respectively.
%A singleton set $\{v\}$ may be written as $v$ and $\{ v_1, v_2, \ldots, v_n \}$ as $v_1 v_2 \cdots v_n$.
The cardinality of a set $W$ is denoted $|W|$.


\subsection{Sum-Product Networks}

A \emph{Sum-Product network} (SPN) $\cal S$ over Boolean variables ${\bf X} = \{X_1,\dots,X_N\}$ is a rooted DAG which contains three types of nodes: indicators, sums and products.
All leaves of $\cal S$ are indicator variables $\lambda_{x_1},\ldots,\lambda_{x_N}$ and $\lambda_{\bar{x}_1},\ldots,\lambda_{\bar{x}_N}$.
All internal nodes are either sum or product.
An indicator variable $\lambda[X=x]$ returns 1 when $X=x$ and 0 otherwise.
Each edge $(v_i,v_j)$ from a sum node $v_i$ has a non-negative weight $w_{ij}$.
The value of a product node is the product of its children.
The value of a sum node is $\sum_{v_j \in Ch{v_i}}{w_{ij}val(v_j)}$.

%
%For $x \in {\bf val(X)}$, we define indicator variables $\lambda_{X=x}(x) := 1(x =x^\prime)$ as input distribution.
%
%graph together with non-negative weights $w_{ij}$ for each edge $(w_i,w_j)$. 

% random variable $X$ with finitely states 
Figure \ref{} shows an SPN over random variables variables $TODO$.
Say that for simplicity, as for Poupart, we are using Boolean variables. 
Show reference for non discrete variables generalization.

\begin{example}
Show example with $P(E=e)$ and indicator variables.	
\end{example}


Definition scope.

Example

Completeness. consistency. Decomposable

There are two method to convert a SPN to a Bayesian Network (BN)\cite{pear88}.


%network polynomial
\subsection{Bayesian Networks}

Definition

Example

Figure

Comment about inference. NP hard.

\subsection{Poupart method}

SPN $\rightarrow$ Normalized SPN $\rightarrow$ Poupart BN

Definition

Algorithm

Comment

\subsection{Peharz}

SPN $\rightarrow$ Augumented SPN $\rightarrow$ Peharz BN

Definition

Algorithm

Comment
\clearpage
\section{On Augmentation of SPNs}
\label{sec:new}


\cite{peharz2015theoretical} states that the BN representation by \cite{zhao2015relationship} can be recovered from the BN representation of augmented SPNs.
This tasks consists of constrain the twin-nodes to be equal to the sum-weights.
By changing the twin wights, all edges between latent variables are removed.



Peharz said: \\
Augmented SPN $\rightarrow$ Poupart BN

I say: \\
Normalized SPN $\rightarrow$ Peharz BN

Current Poupart:\\
	SPN $\rightarrow$ Normalized SPN $\rightarrow$ Poupart BN

We know:\\
	Augumented SPN $=$ Normalized SPN $+$ Twin Nodes
	

\subsection{Introduce Twin Nodes in Normalized SPN}

Poupart:\\
	SPN $\rightarrow$ Normalized SPN + Twin Nodes (Augmented SPN)$\rightarrow$ Poupart BN
	
Use Peharz method to:\\
	Nirmalized SPN $\rightarrow$ Augmented SPN $\rightarrow$ Pehartz BN
	
\begin{algorithm}[!ht]
    \caption{Build BN Structure from a SPN.}
    \label{alg:poupart}
    \begin{algorithmic}[1]
%        \Procedure{CircuitPropagation}{${\cal AC}$,$\upsilon()$,$d()$}
        \item[\textbf{Input:} (augmented) normal SPN $\cal S$]
%        \Statex{Normal SPN $\cal S$}
        \item[\textbf{Output:} BN ${\cal B} = ({\cal B}_V, {\cal B}_E)$]
%		\item[\textbf{Main:}]
		\State $R \leftarrow$ root of ${\cal S}$
		\If{$R$ is a terminal node over variable $X$}
			\State Create an observable variable $X$
			\State ${\cal B}_V \leftarrow {\cal B}_V \cup X$
		\Else
			\For{each child $R_i$ of $R$}
				\If{BN has not been built for ${\cal S}_{R_i}$}
					\State Recursively build BN Structure for ${\cal S}_{R_i}$
				\EndIf
			\EndFor
			\If{$R$ is a sum node}
				\State Create a hidden variable $H_R$ associated with $R$
				\State ${\cal B}_V \leftarrow {\cal B}_V \cup \{H_R\}$
				\For{each observable variable $X \in {\cal S}_R$}
					\State ${\cal B}_E \leftarrow {\cal B}_E \cup \{(H_R,X)\}$
				\EndFor
			\EndIf
		\EndIf		
%    \EndProcedure
    \end{algorithmic}
\end{algorithm}
	
\subsection{Alternative for Twin Nodes}

Instead of introducing twin nodes in Normalized SPN we can modify cope definition in normalized SPN.

The method consists in creating a storage for scope in the nodes.
in the Pehartz algorithm we would add the missing variables to the scope of some nodes.
\section{Analysis}
\label{sec:analysis}


{\bf Upgrade Poupart method for encoding LV interpretation.}

[Poon11] said SPNs are mixtures models

Mixture models $\Rightarrow$ LV interpretation of SPNs

Normalized SPNs are equivalent to SPNs

Normalized SPNs + LV introduction $\Rightarrow$ completeness problem

New method fix with Pehartz solution


This new method allows to a initiative talk apply Poupart to convert a much larger class of BNs to SPn
\input{sections/conclusion}

% References
\bibliographystyle{splncs03}
\bibliography{references/references}

\end{document}
